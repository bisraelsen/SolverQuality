\documentclass[sigconf,authordraft]{acmart}

\usepackage{pgfgantt}
\usepackage{booktabs}
\usepackage{graphicx,caption}
\usepackage{mathtools}%loads amsmath
\usepackage{amssymb,amsfonts}
% \usepackage[caption=false]{subfig}
% \captionsetup{width=\textwidth}
\usepackage{enumitem}
\usepackage{hyperref}
\usepackage[margin=0.75in]{geometry}

% Copyright
%\setcopyright{none}
%\setcopyright{acmcopyright}
%\setcopyright{acmlicensed}
\setcopyright{rightsretained}
%\setcopyright{usgov}
%\setcopyright{usgovmixed}
%\setcopyright{cagov}
%\setcopyright{cagovmixed}


% DOI
\acmDOI{10.475/123_4}

% ISBN
\acmISBN{123-4567-24-567/08/06}

%Conference
\acmConference[FAT* 2019]{Conference on Fairness, Accountability, and Tranparency}{January 2019}{Atlanta, Georgia USA}
\acmYear{2019}
\copyrightyear{2019}

\acmPrice{15.00}

\acmSubmissionID{123-A12-B3}

%%%%% shortcut commands
\newcommand{\solve}{$\mathcal{S}$}
\newcommand{\solvestar}{$\mathcal{S}^*$}
\newcommand{\task}{$\mathcal{T}$}
\newcommand{\taski}{$\mathcal{T}_i$}
\newcommand{\taskN}{$\mathcal{T}_N$}
\newcommand{\rwd}{$\mathcal{R}$}
\newcommand{\rwdstar}{$\mathcal{R}^*$}
\newcommand{\rwdstari}{$\mathcal{R}^*_i$}
\newcommand{\rwdstarN}{$\mathcal{R}^*_N$}
\newcommand{\rwdstarapprox}{$\widetilde{\mathcal{R}}^*$}
\newcommand{\rwdstariapprox}{$\widetilde{\mathcal{R}}^*_i$}
\newcommand{\policy}{$\mathcal{\pi}$}
\newcommand{\policystar}{$\mathcal{\pi}^*$}
\newcommand{\surrogate}{$\mathcal{M}^*(\mathcal{T})$}
\newcommand{\xQ}{$x_Q$}
\newcommand{\xP}{$x_P$}
\newcommand{\xH}{$x_H$}
\newcommand{\xI}{$x_I$}
\newcommand{\xM}{$x_M$}
%%%%%%%%%%
\newcommand{\hlr}[1]{{\color{red} #1}}
\newcommand{\hlb}[1]{{\color{blue} #1}}
\newcommand{\hlo}[1]{{\color{orange} #1}}
\newcommand{\nisar}[1]{\hlr{NRA: #1}}
\newcommand{\brett}[1]{\hlo{BWI: #1}}

%%%%%%%%%%

\begin{document}

\title{Self-Confidence of Autonomous Systems Via Meta-Analysis of Internal Processes}
\author{Author 1}
    \orcid{0000-0000-0000-0000}
    \email{ano.ny@mous.com}
\author{Author 2}
    \affiliation{%
        \institution{Prestigious Institution}
        \city{city}
        \state{state}
        \country{country}
    }
\begin{abstract}
    Those who use advanced autonomous systems need to be able to understand their competencies and limitations in order to trust and use them appropriately. It is possible for these systems to influence a user's trust via `algorithmic assurances'. Designing advanced autonomous systems with the capacity of assessing their own capabilities is one approach to creating an algorithmic assurance. The idea of `self-confidence' of autonomous systems is introduced; one component of self-confidence, called `solver-quality' is discussed in the context of Markov decision processes for autonomous systems. Markov decision processes underlie much of the theory of reinforcement learning, and are commonly used for planning and decision making under uncertainty in robotics and autonomous systems. The solver-quality metric is formally defined, and derived; several computational trials indicate that the metric exhibits the desired properties. Discussion of results, and avenues for future investigation are included.
\end{abstract}
\maketitle

\input{"introduction.tex"}
\input{"methodology.tex"}
\input{"results.tex"}
\input{"conclusions.tex"}

\bibliographystyle{ACM-Reference-Format}
\bibliography{References}

\end{document}
